% Created 2020-11-28 Sat 18:43
% Intended LaTeX compiler: pdflatex
\documentclass[11pt]{article}
\usepackage[utf8]{inputenc}
\usepackage[T1]{fontenc}
\usepackage{graphicx}
\usepackage{grffile}
\usepackage{longtable}
\usepackage{wrapfig}
\usepackage{rotating}
\usepackage[normalem]{ulem}
\usepackage{amsmath}
\usepackage{textcomp}
\usepackage{amssymb}
\usepackage{capt-of}
\usepackage{hyperref}
\usepackage{tikz}
\usepackage[newfloat]{minted}
\author{Gustavo Puche}
\date{\today}
\title{Emacs doxy-graph minor mode}
\hypersetup{
 pdfauthor={Gustavo Puche},
 pdftitle={Emacs doxy-graph minor mode},
 pdfkeywords={},
 pdfsubject={},
 pdfcreator={Emacs 27.0.50 (Org mode 9.3.8)}, 
 pdflang={English}}
\begin{document}

\maketitle
\tableofcontents

\texttt{doxy-graph-mode} links source code to doxygen call graphs.

It allows to interactively see the call graph or the inverted call
graph of a given function or method from source code.

\section{Requirements}
\label{sec:orge533f33}

\begin{itemize}
\item emacs 26.3 (or newer).
\item doxygen 1.8.13 (or newer).
\end{itemize}

\section{Basic Setup}
\label{sec:org3e8d4e1}

After getting doxy-graph-mode.el you can add some of this lines to your \texttt{.emacs} file
if you want that doxy-graph-mode auto-loads when c-mode, c++-mode or python-mode are enabled.

\begin{minted}[breaklines,fontsize=\small,tabsize=2,frame=lines,autogobble]{elisp}
(require 'doxy-graph-mode)

(add-hook 'c-mode-hook 'doxy-graph-mode)
(add-hook 'c++-mode-hook 'doxy-graph-mode)
(add-hook 'python-mode-hook 'doxy-graph-mode)
\end{minted}

You can add a hook for another one of the languages supported by doxygen.

\section{Usage}
\label{sec:orgbaf6fa3}

Firts of all you have to generate doxygen latex documentation
including call graphs. To carry out this you can generate a \texttt{Doxyfile}
config file with the command \texttt{doxygen -g} in your source code path.

It is necessary to change the following settings inside \texttt{Doxyfile}.

\begin{minted}[breaklines,fontsize=\small,tabsize=2,frame=lines,autogobble]{bash}
CALL_GRAPH             = YES

CALLER_GRAPH           = YES
\end{minted}

An example of \texttt{Doxyfile} is given in this repository.

You can invoque \texttt{doxy-graph-mode} commad \texttt{M x doxy-graph-mode}.

When mini mode is active you can position the cursor over function
name and visualize call graph and inverted call graph.

\begin{center}
\includegraphics[width=.9\linewidth]{./img/call-graph.png}
\end{center}

Default keybindings are:

\begin{itemize}
\item \texttt{<C-f1>} Opens direct call graph.
\item \texttt{<C-f2>} Opens inverted call graph.
\end{itemize}

\begin{center}
\includegraphics[width=.9\linewidth]{./img/inverted-call-graph.png}
\end{center}

First time is called \texttt{doxy-graph-open-call-graph} or
\texttt{doxy-graph-open-inverted-call-graph} it is asked for select doxygen
latex folder. You can choose doxygen latex folder using the directory
chooser.

Calling the interactive function \texttt{doxy-graph-set-latex-path} you can
change the folder where doxygen latex docs are placed.

If you wish you can change default keybindings using
\texttt{doxy-graph-mode-map}. The following example binds \texttt{C c c} and \texttt{C c i}
like default shortcuts.

\begin{minted}[breaklines,fontsize=\small,tabsize=2,frame=lines,autogobble]{elisp}
(define-key doxy-graph-mode-map (kbd "C-c c") 'doxy-graph-open-call-graph)
(define-key doxy-graph-mode-map (kbd "C-c i") 'doxy-graph-open-inverted-call-graph)
\end{minted}
\end{document}
